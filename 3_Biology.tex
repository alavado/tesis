\chapter{El \textit{skeleton} en biología}
\label{ch:metrics}

Según lo señalado en el Capítulo \ref{ch:introduction}, toda comparación de algoritmos de cálculo del \textit{skeleton} debe hacerse en cuanto al uso del \textit{skeleton} en alguna aplicación específica. Este capítulo es una revisión de la aplicación del \textit{skeleton} que motiva este trabajo de tesis.



Sacar la justificación de aquí:
Neuromantic – from semi-manual to semi-automatic reconstruction of neuron morphology
salen los tipos de trazados y blabla, el que interesa es el automatico

Simple Neurite tracer \cite{longair2011simple} es el de ImageJ asistido, donde se seleccionan los start y end points y advina el camino -esto no importa mucho

skeleton es parte central del proceso, porque despues se analiza con analize skel


en neurite tracer tambien skeletonizan sin decir como



\cite{schmitz2011automated} To calculate dendrite length, the dendrite mask is skeletonized
using MATLAB’s built in bwmorph function to obtain a single-pixel
representation


\cite{ho2011neurphologyj} 2D propone un método basado en algoritmos misteriosos de ImageJ, incluyendo misterioso metodo "skeletonize", y se compara contra simple Neurite tracer - es bueno que incluye un resumen de los paquetes comerciales

\cite{pani2014morphoneuronet} "was skeletonized"

\cite{billeci2013neuronmorphological} The techniques used for skeletonization are standard and therefore not described in detail



Usan una implementación del algoritmo de thinning de \cite{lee1994building} en 2d y 3d, que es parecido al de palagyi pero con arbol de decision.

Además en algunos papers se comparan con otros algoritmos, indicando las cosas que importan, como el largo promedio de las neuritas. sacar de aqui la justificacion

\subsection{Métricas de comparación}

El \textit{skeleton} puede verse como un grafo no dirigido \cite{bai2008path}. Tomando esta idea, propondremos métricas provenientes en su mayoría de la teoría de grafos. Separaremos los objetos en dos tipos: objetos con túneles \cite{svensson2003finding} y objetos sin túneles. Las métricas dependerán del tipo de objeto, notando que el \textit{skeleton} de un objeto con túneles y el de uno sin túneles se parecen a un grafo cíclico y a un árbol respectivamente.

\subsubsection{Métricas para objetos con túneles}

\begin{itemize}
\item Número de nodos
\item Número de arcos
\item Número de ciclos
\item Distribución de longitud de arcos
\item Volúmen \footnote{Número de vóxeles del \textit{skeleton}}
\end{itemize}

\subsubsection{Métricas para objetos sin túneles}

\begin{itemize}
\item Número de nodos
\item Número de arcos
\item Largo del camino máximo
\item Distribución de longitud de arcos
\item Distribución de ángulos de bifurcación
\item Asimetría \footnote{Valor entre 0 y 1 que cuantifica el balance de un árbol. Vale 0 para un árbol perfectamente balanceado y 1 para un árbol con todos sus subárboles completamente desbalanceados. Se define como:
\[
A = \frac{1}{N}\sum A_{p}(r_{izq},r_{der})
\]
Esto corresponde a calcular el promedio de las asimetrías de todos los subárboles del \textit{skeleton} \cite{van2004morphological}.}
\item Volúmen
\end{itemize}

\subsection{Datos}

No se cuenta con objetos con \textit{skeletons} conocidos. por lo que generaremos objetos con y sin túneles para medir el desempeño de los algoritmos. Como se ha señalado anteriormente, también se contempla una verificación con expertos para validar las métricas al ser calculadas en \textit{skeletons} extraídos desde objetos reales.

\subsubsection{Datos Generados}

En SCIAN-Lab se han implementado generadores de \textit{skeletons} con y sin ciclos \cite{villarroel2012}. De este modo, para estos datos se cuenta con la \textit{ground truth} para determinar el error que cada algoritmo tiene en cada métrica.

\subsubsection{Datos Reales}

SCIAN-Lab cuenta con decenas de volúmenes y mallas geométricas de estructuras biológicas correctamente segmentadas. No obstante, el \textit{skeleton} de estos objetos es desconocido, por lo que la validación de las métricas requerirá la evaluación por parte de expertos en el área, a través de, por ejemplo, un cuestionario.

También será ilustrativo calcular las métricas para objetos de las principales bases de datos de uso público comúnmente usadas en este problema, notablemente la Princeton Shape Benchmark Database \cite{shilane2004princeton}.

\section{Tabla de resultados}

\begin{center}
    \begin{tabular}{ | l | c | c |}
    \hline
    Algoritmo & Número de nodos & Número de arcos \\ \hline
    Palàgyi et al.& 100 & 200 \\ \hline
    Arcelli et al.& 100 & 200 \\ \hline
    Siddiqi et al.& 100 & 200 \\ \hline
    Au et al.& 100 & 200 \\ \hline
    \end{tabular}
\end{center}